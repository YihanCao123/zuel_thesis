\chapter{图表及引用}

\section{图片}

\LaTeX 支持多种图片格式,并且可以通过label整理出非常整洁的图索引。当然,也可以使用tikz等工具直接在\LaTeX 中完成矢量图的绘制。一般插入外部图片,可以使用:

\begin{figure}[H]
\centering
\includegraphics [width=0.8\textwidth]{figure//logo.jpg}
\caption{中南财经政法大学校徽}\label{fig1}
\end{figure}

这里可以设置多种参数,具体请见latex教程。

\section{表格与列表}

当用\LaTeX 编辑长表格时常常会很麻烦,这时可以使用在线工具,或者使用excel的插件,这些都可以讲excel表格转换为latex表格。
一个普通的三线表如下:

\begin{table}[H]
	\centering
	\caption{三线表格}
	\label{tab1}
	\begin{tabular}{lcc}
\hline
1 & 2 & 3\\
\hline
4 & 5 & 6\\
7 & 8 & 9\\
\hline
	\end{tabular}
\end{table}

\section{无序列表与有序列表}

\LaTeX 支持有序与无序的列表,有序列表为:
\begin{enumerate}
\item{这是一个有序列表。}
\item{这是两个有序列表。}
\end{enumerate}

无序列表为:
\begin{itemize}
\item{这是一个无序列表。}
\item{这是一个无序列表。}
\end{itemize}

\section{引用设置}

\subsection{脚注与超链接}

使用脚注,常常是在文中出现了来自网页等来源的信息时。在\LaTeX 中插入脚注的方法为\footnote{在这里插入脚注}。

脚注有时会链接到网址,latex中同样可以插入网址,例如:\href{https://github.com}{github}。

使用ref可以链接到同一篇文章中的位置,例如链接到前面的三线表格~\ref{tab1}

\subsection{参考文献}

参考文献可以使用cite来标注。使用逗号分隔可以同时引用多个参考文献。注意,这里参考文献脚注出现的位置是作者定义的,使用者可以在config中更改。

可以直接在文档末尾手打引用文献,但是这种方式并不便捷,我们推荐使用bibtex的模式。在参考文献界面(例如researchgate等网站)上导出bib格式的参考文献,并复制入ref/reference.bib中,就可以在文中直接生成参考文献\cite{article1}。

使用bibTex编译参考文献,需要三个步骤: 
\begin{itemize}
\item{首先,使用Xelatex编译}
\item{使用BibTex编译一次}
\item{使用Xelatex编译两次}
\end{itemize}

